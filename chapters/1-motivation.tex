\documentclass[../Lebensziel.tex]{subfiles}
\graphicspath{{\subfix{../img/}}}
\begin{document}

\chapter{Motivation}\thispagestyle{fancy}

Ein häufiges Problem auf das früher oder später jeder stößt, ist die Frage nach dem Sinn des Lebens.

Grundlegend kann man behaupten, dass es nicht DEN Sinn des Lebens gibt. Er ist viel subjektiver und invidueller. Daher ist der Weg dorthin für jede Person ein anderer. Dennoch gibt es einige Hilfsmittel und Methoden, wie man sich seinen Sinn im Leben klar werden kann. Diese sollen hier erläutert und mit Arbeitsblättern als Unterstützung und kleiner Leitfaden gelten.

Der Sinn des Lebens wird aus unterschiedlichen Zielen gewählt, die man in seinem Leben zu verfolgen gedenkt und mit Motivation und Präzision erreichen kann. Er wird also zum Lebensziel.
\vspace{.5cm}

Um dein Lebensziel zu finden, gehe ähnlich zu einem großen Projekt vor. Am Anfang steht der Entwurf des Projektsplans, um überhaupt zu wissen, wo man steht und wie man vorgeht. Dazu müssen viele einzelne Fragen zu vielen unterschiedlichen Themen oder Kategorien geklärt werden. Und auch wenn man alleine, an einem solchen Plan arbeitet sollte man alles aufschreiben, um nicht die Übersicht zu verlieren. Betrachte dein Leben und deine Lebensweise als ein Projekt, in dem deine Ziele die Projektprioritäten sind und die Planung deinen Weg dorthin gestaltet.
Verfalle dabei nicht in eine strikte unflexible Herangehensweise. Pläne ändern sich, genauso wie die Umstände, die unvorhersehbar Eintreten können.
\vspace{.5cm}

Diese kleine Orientierungshilfe stützt sich auf bekannte Werke wie ,,12 Rules for Life''\cite{rules-for-life} von Jordan Peterson, ,,Cafe am Rande der Welt''\cite{cafe-rande-welt} von John Strelecky und ,,Leben ist Leidenschaft''\cite{leben-leidenschaft} von Matthias Kirchner. Ich kann jedem, der sich weiter mit dem Thema auseinandersetzten will, diese nur wärmstens Empfehlen.

\end{document}