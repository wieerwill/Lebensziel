\documentclass[../Lebensziel.tex]{subfiles}
\graphicspath{{\subfix{../img/}}}
\begin{document}

\chapter{Wer bin ich?}\thispagestyle{fancy}
Die Frage ,,Wer bin ich?'' ist kaum zu beantworten und hat schon manche in den Wahnsinn getrieben. Und das hat einen Grund: Es ist eine absolut unzulängliche Frage, die nicht genau genug gestellt wird. Hinter dieser einen Frage stecken weitaus mehr und diversere Fragestellungen, die man herausfiltern muss. Beispielsweise:
\begin{itemize}
    \item Welcher Typ Mensch bin ich?
    \item Was sind meine Stärken?
    \item Was sind meine Schwächen?
    \item Welche Wünsche habe ich und wo möchte ich hin?
    \item Wo stehe ich im Vergleich zu anderen?
    \item Was ist mir wichtig?
    \item Was will ich mit meinem Leben anfangen?
\end{itemize}
Diese einzelnen Fragen sind wesentlich einfacher zu beantworten.
Stellt man eine Frage spezifisch genug, enthält diese meist schon einen Großteil der Antwort.
Also stelle deine Fragen detaillierter.

\newpage
\section{Grundfragen zum eigenen Selbst}
Es gibt Fragen die sich nur wenige Menschen stellen. Wir gehen hier auf die häufigeren weiter ein. Nach dem selben vorgehen kannst du dir deine eigenen Fragen die nur für dich aufkommen beantworten und dir Gedanken machen, wo du hin willst.

\subsection*{Was sind Deine Stärken? Was sind Deine Schwächen?}
Die Frage wer Du wirklich bist fragt auch immer, was Dich von anderen unterscheidet. Was macht Dich einzigartig? Diese Idee führt zu einem groben Vergleich, von deinen Stärken und Schwächen gegenüber dem, was du für normal hältst.

Nimm ein Blatt Papier und teile es in zwei Spalten auf. Auf die linke Seite schreibst Du Deine Stärken und auf der rechten gegenüberliegenden Seite deine Schwächen, mit denen du dich unsicher fühlst. Eine kleine Vorlage\ref{stärken+schwächen} ist gleich weiter unten für dich bereit. Natürlich kannst du deine eigene Tabelle mit mehr Spalten nutzen

Vielen Menschen fällt es leichter, sich selber zu kritisieren statt zu loben. Versuche daher mindestens so viele Stärken wie Schwächen aufzulisten.

Beispiele können sein:
\begin{multicols}{2}
    \begin{itemize}
        \item Ich bin gut in Mathe
        \item Ich kann hervorragend Kochen
        \item Ich kann Gitarre spielen
        \item Ich bin ein guter Zuhörer
              %\item Ich kann mich gut in andere Menschen einfühlen
              %\item Ich bin ehrlich
              %\item Ich bin gewissenhaft
              %\item Ich halte mein Wort
              %\item Ich kann gut zeichnen
        \item Ich bin sportlich
        \item Ich bin kreativ
    \end{itemize}
\end{multicols}

\begin{table}[h!]
    \centering
    \setlength{\tabcolsep}{18pt}
    \renewcommand{\arraystretch}{1.5}
    \begin{tabular}{p{5.5cm}|p{5.5cm}}
        \textbf{Meine Stärken} & \textbf{Meine Schwächen} \\\hline
        1.                     & 1.                       \\\hline
        2.                     & 2.                       \\\hline
        3.                     & 3.                       \\\hline
        4.                     & 4.                       \\\hline
        5.                     & 5.                       \\\hline
        6.                     & 6.                       \\\hline
        7.                     & 7.
    \end{tabular}
    %\caption{Deine Stärken und Schwächen}
    \label{stärken+schwächen}
\end{table}

\subsection*{Welche Werte möchte ich leben?}
Welche Werte sind Dir wichtig?
Werte sind Grundeigenschaften wie Ehrlichkeit, Aufrichtigkeit, Zielstrebigkeit und Offenheit, die Du unabhängig der einzelnen Situation leben möchtest.
Welche Werte möchtest Du in Deinem Handeln zum Ausdruck bringen?

\begin{table}[h!]
    \centering
    \setlength{\tabcolsep}{18pt}
    \renewcommand{\arraystretch}{2}
    \begin{tabular}{p{12cm}}
        \textbf{Meine Werte} \\\hline
        1.                   \\\hline
        2.                   \\\hline
        3.                   \\\hline
        4.                   \\\hline
        5.
    \end{tabular}
    %\caption{Deine Lebens-Werte}
    \label{werte-leben}
\end{table}

Auf die Werte werden wir später im Kapitel Persönlichkeit weiter eingehen und deine Liste verfeinern.

\subsection*{Was gibt Dir Energie, was raubt Dir Energie?}
Gehe gedanklich durch Deinen Alltag. Was gibt Dir Energie?
Der gemütliche Kaffee am Morgen? Gute Gespräche mit Freunden? Sport am morgen oder nach der Arbeit? Einfach Zeit für Dich?

Mache dir eine Liste, was Dir Kraft und Energie gibt. Was Dir Kraft gibt ist das, was gut zu Dir passt.
Genauso das Gegenteil. Was raubt Dir Energie?
\begin{itemize}
    \item Menschen, die negativ über andere sprechen?
    \item Dich morgens früh zu Deinem Job raus quälen?
    \item Langweilige Gespräche?
    \item Große Menschenmassen?
    \item Eine unaufgeräumte Wohnung?
\end{itemize}

Lass Deinen Gedanken freien Lauf.

\begin{table}[h!]
    \centering
    \setlength{\tabcolsep}{18pt}
    \renewcommand{\arraystretch}{1.5}
    \begin{tabular}{p{5.5cm}|p{5.5cm}}
        \textbf{Gibt mir Engergie} & \textbf{Raubt mir Energie} \\\hline
        1.                         & 1.                         \\\hline
        2.                         & 2.                         \\\hline
        3.                         & 3.                         \\\hline
        4.                         & 4.                         \\\hline
        5.                         & 5.                         \\\hline
        6.                         & 6.                         \\\hline
        7.                         & 7.                         \\\hline
        8.                         & 8.                         \\\hline
        9.                         & 9.                         \\\hline
        10.                        & 10.
    \end{tabular}
    %\caption{Hole dir Energie}
    \label{energie}
\end{table}

\subsection*{Was sagen andere über mich?}
Wie würden Deine Freunde Dich beschreiben? Welche Eigenschaften schreiben sie Dir zu und schätzen sie besonders an Dir?
Überlege und schreibe dir auf, wie Deine besten Freunde Dich beschreiben würden. Welche Kritik üben sie an dir und welche Seite finden sie an dir besonders toll?

Mache dir zuerst selbst Gedanken, was du in den letzten Tagen oder Wochen mitbekommen hast und trage es in Tabelle \ref{selbstbild} ein

\begin{table}[h!]
    \centering
    \setlength{\tabcolsep}{18pt}
    \renewcommand{\arraystretch}{1.5}
    \begin{tabular}{p{3.3cm}|p{3.3cm}|p{3.3cm}}
        \textbf{Was bin ich?} & \textbf{Was will ich?} & \textbf{Was habe ich?} \\\hline
        1.                    & 1.                     & 1.                     \\\hline
        2.                    & 2.                     & 2.                     \\\hline
        3.                    & 3.                     & 3.                     \\\hline
        4.                    & 4.                     & 4.                     \\\hline
        5.                    & 5.                     & 5.                     \\\hline
        6.                    & 6.                     & 6.                     \\\hline
        7.                    & 7.                     & 7.                     \\\hline
        8.                    & 8.                     & 8.                     \\\hline
        9.                    & 9.                     & 9.                     \\\hline
        10.                   & 10.                    & 10.
    \end{tabular}
    \caption{Dein Selbstbild}
    \label{selbstbild}
\end{table}

Das Spannende ist, dass unser Selbstbild oft nur wenig mit dem übereinstimmt, wie andere uns sehen.
Es ist daher extrem wertvoll, Dein Selbstbild mit Deinem Fremdbild abzugleichen. Frage deine Freunde direkt und mach Dich auf einige Überraschungen gefasst. Wichtig hierbei ist, dass deine Freunde unbeeinflusst von deinem Selbstbild sind.
Erzähle Ihnen erst danach, wie du dich selbst siehst.
\newpage
\begin{table}[h!]
    \centering
    \setlength{\tabcolsep}{18pt}
    \renewcommand{\arraystretch}{1.5}
    \begin{tabular}{p{3.3cm}|p{3.3cm}|p{3.3cm}}
        \textbf{Was bin ich?} & \textbf{Was will ich?} & \textbf{Was habe ich?} \\\hline
        1.                    & 1.                     & 1.                     \\\hline
        2.                    & 2.                     & 2.                     \\\hline
        3.                    & 3.                     & 3.                     \\\hline
        4.                    & 4.                     & 4.                     \\\hline
        5.                    & 5.                     & 5.                     \\\hline
        6.                    & 6.                     & 6.                     \\\hline
        7.                    & 7.                     & 7.                     \\\hline
        8.                    & 8.                     & 8.                     \\\hline
        9.                    & 9.                     & 9.                     \\\hline
        10.                   & 10.                    & 10.
    \end{tabular}
    \caption{Dein Fremdbild}
    \label{fremdbild}
\end{table}
\newpage

\subsection*{Was sind Deine Ängste, was sind Deine Wünsche.}
Was wünschst Du Dir? Was fürchtest Du? Ängste können Wünsche antreiben oder blockieren. Sei dir klar darüber, welchen Ängsten du dich stellen musst um dein Ziel zu erreichen und mit welchen Ängsten du befreit leben kannst.

Nimm dir etwas Zeit um dir darüber Gedanken zu machen. Nach kurzer Zeit wirst Du ein viel klareres Bild davon haben, was Dich antreibt und anzieht.

Nimm Dir jeden Tag 5 Minuten Zeit, um darüber nachzudenken, was sich bei deinen Ängsten und Wünschen geändert hat. Und besonders: Was hat diese Änderung hervorgerufen?

\begin{table}[h!]
    \centering
    \setlength{\tabcolsep}{18pt}
    \renewcommand{\arraystretch}{1.5}
    \begin{tabular}{p{5.5cm}|p{5.5cm}}
        \textbf{Ängste} & \textbf{Wünsche} \\\hline
        1.              & 1.               \\\hline
        2.              & 2.               \\\hline
        3.              & 3.               \\\hline
        4.              & 4.               \\\hline
        5.              & 5.               \\\hline
        6.              & 6.               \\\hline
        7.              & 7.               \\\hline
        8.              & 8.               \\\hline
        9.              & 9.               \\\hline
        10.             & 10.
    \end{tabular}
    %\caption{angst+wunsch}
    \label{angst+wunsch}
\end{table}

\newpage
\subsection*{Was am Ende wirklich zählt}
Vielleicht kennst du das: Ich weiß manchmal nicht mehr, was mir wirklich wichtig ist. Man verliert sich in Alltagsproblemen, die Gedanken werden Stumpf und die eigenen Prioritäten verschwimmen. Wir werden versuchen, dieses Problem mit Lebenszielen entgegenzuwirken und dir deinen Antrieb jederzeit bewusst zu sein.

Stelle Dir vor, Du liegst auf Deinem Sterbebett. Du weißt ,,Das wars, von hier werde ich nie wieder aufstehen.''
Blicke jetzt auf Dein Leben zurück.
\begin{itemize}
    \item Welche Entscheidungen erfüllen Dich mit Stolz?
    \item Welche Entscheidungen bereust Du?
    \item Welchen Lebenspfad hast du gewählt?
\end{itemize}

Blicke von Deinem Sterbebett zurück auf Dein jetziges Leben. Was würdest Du Dir selber raten?
(Das hier ist die Frage nach Deinen Zielen im Leben.)

\begin{table}[h!]
    \centering
    \setlength{\tabcolsep}{18pt}
    \renewcommand{\arraystretch}{2}
    \begin{tabular}{p{3.3cm}|p{3.3cm}|p{3.3cm}}
        Was erfüllt mich? & Was bereue ich? & Was rate ich mir? \\\hline
        1.                & 1.              & 1.                \\\hline
        2.                & 2.              & 2.                \\\hline
        3.                & 3.              & 3.                \\\hline
        4.                & 4.              & 4.                \\\hline
        5.                & 5.              & 5.                \\\hline
        6.                & 6.              & 6.                \\\hline
        7.                & 7.              & 7.                \\\hline
        8.                & 8.              & 8.                \\\hline
        9.                & 9.              & 9.
    \end{tabular}
    %\caption{Sterbebett}
    \label{sterbebett}
\end{table}

\subsection*{Was liebst Du und was hasst Du}
Gleiches Spiel wie vorhin.
Zwei Spalten auf Deinem Blatt. Auf die linke Seite kommen all die Dinge, die Du liebst. Auf die rechte Seite all die Dinge, die Du nicht leiden kannst.

\begin{table}[h!]
    \centering
    \setlength{\tabcolsep}{18pt}
    \renewcommand{\arraystretch}{1.5}
    \begin{tabular}{p{5.5cm}|p{5.5cm}}
        \textbf{Liebe ich} & \textbf{Hasse ich} \\\hline
        1.                 & 1.                 \\\hline
        2.                 & 2.                 \\\hline
        3.                 & 3.                 \\\hline
        4.                 & 4.                 \\\hline
        5.                 & 5.                 \\\hline
        6.                 & 6.                 \\\hline
        7.                 & 7.                 \\\hline
        8.                 & 8.                 \\\hline
        9.                 & 9.                 \\\hline
        10.                & 10.                \\\hline
        11.                & 11.                \\\hline
        12.                & 12.                \\\hline
        13.                & 13.                \\\hline
        14.                & 14.                \\\hline
        15.                & 15.                \\\hline
        16.                & 16.                \\\hline
        17.                & 17.                \\\hline
        18.                & 18.                \\\hline
        19.                & 19.                \\\hline
        20.                & 20.
    \end{tabular}
    %\caption{Liebe und Hass}
    \label{liebe+hass}
\end{table}

\newpage
\subsection*{Was ist Deine Lebensvision?}
Du hast dir bereits einige Gedanken gemacht und über dein Leben sinniert. Stelle dir jetzt die Frage: Wohin willst Du in Deinem Leben unterwegs sein? Was sind deine Ziele die Du Dir setzt? Welcher Mensch Du bist? Schaffe Klarheit über den Sinn Deines Lebens!

Für den Moment kannst du dir deine Ziele kurz und knapp notieren. Die Tabelle muss dabei nicht voll werden, weniger Ziele heißt mehr Konzentration auf wenige Punkte.

\begin{table}[h!]
    \centering
    \setlength{\tabcolsep}{18pt}
    \renewcommand{\arraystretch}{1.5}
    \begin{tabular}{p{5.5cm}|p{5.5cm}}
        \textbf{Meine Ziele} & \textbf{Das bin ich} \\\hline
        1.                   & 1.                   \\\hline
        2.                   & 2.                   \\\hline
        3.                   & 3.                   \\\hline
        4.                   & 4.                   \\\hline
        5.                   & 5.                   \\\hline
        6.                   & 6.                   \\\hline
        7.                   & 7.                   \\\hline
        8.                   & 8.                   \\\hline
        9.                   & 9.                   \\\hline
        10.                  & 10.
    \end{tabular}
    %\caption{kleines Lebensziel}
    \label{kleines Lebensziel}
\end{table}

Am besten erstellst Du dazu später ein Vision Board, dieses wird im Kapitel ,,Vision Board'' genauer erklärt.

\end{document}
