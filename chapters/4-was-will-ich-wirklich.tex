\documentclass[../Lebensziel.tex]{subfiles}
\graphicspath{{\subfix{../img/}}}
\begin{document}

\chapter{Was will ich wirklich?}\thispagestyle{fancy}

\section{Der Grund warum Du nicht weißt was Du willst}
Es ist verdammt einfach, nicht zu wissen was man will.
Der Grund dafür ist, dass jeder Mensch Gefühl und Verstand besitzt.
Der Verstand ist für die langfristige Planung verantwortlich.
Das Gefühl lebt im hier und jetzt.

Gefühl und Verstand sind ein geniales Team. Nur nicht immer einig, in dem was sie wollen.
\begin{itemize}
    \item Gehe ich ins Bett weil ich morgen fit sein möchte, oder schaue ich noch eine Folge Game of Thrones?
    \item Stehe ich um 6.00 Uhr auf und jogge, oder bleibe ich im warmen Bett?
    \item Schreibe ich meine Hausarbeit fertig, oder gehe ich mit Freunden feiern?
    \item Nehme ich den Job an, um auf eigenen Beinen zu stehen oder genieße ich noch länger meine Freizeit?
    \item Ziehe ich nach Hamburg wo meine Freunde sind, oder nach München an die bessere Uni?
\end{itemize}

Wenn Verstand und Gefühl sich widersprechen, ist Chaos vorprogrammiert. Dann bleibst du ratlos zurück und fragst dich: ,,Was will ich denn jetzt wirklich?''

\section{Voraussetzungen, um zu wissen was Du willst}
Um genau zu wissen, was Du willst, müssen 5 Voraussetzungen erfüllt sein.
\begin{description}
    \item[Du kennst dein langfristiges Ziel] Woher willst du wissen, was du hier und heute möchtest, wenn du noch nicht einmal die grobe Richtung in deinem Leben kennst? Dein Ziel \ref{kleines Lebensziel} solltest du im vorherigen Kapitel bereits gefunden haben
    \item[Du kennst deine persönlichen Werte] Deine persönlichen Werte entscheiden, wie du dich verhalten möchtest. Selbst wenn du nicht weißt was du willst, kannst du immer entscheiden, welche Option eher deinen Werten entspricht. Auch im letzten Kapitel wurde deine Werte \ref{persönliche-werte} schon gesucht.
    \item[Dir ist klar, was dir Freude macht und Frust bereitet] Was will ich wirklich? Das was mir Freude und Sinn bereitet. Schaue in deine Listen \ref{energie} und \ref{liebe+hass} vom letzten Kapitel.
    \item[Du hast genug Informationen, um eine Entscheidung zu treffen] Manchmal haben wir einfach nicht genug Informationen, um mit gutem Gefühl eine Entscheidung zu treffen. Damit musst Du leben. Da hilft nur der Sprung ins kalte Wasser.
    \item[Herz und Kopf wollen das Gleiche] Ist das nicht der Fall fühlst du dich innerlich zerrissen. Im Zweifel ist es besser, dem eigenen Gefühl zu folgen. Denn wenn es schief geht, weißt du wenigstens, dass du dir treu geblieben bist.
\end{description}

Damit Du eine klares ,,Ja'' oder ,,Nein'' spürst, müssen alle Voraussetzungen erfüllt sein.

\subsection{Sammele mehr Informationen}
Du kannst nur klar entscheiden was du willst, wenn du genug Informationen für diese Entscheidung hast. Der häufigste Fall, warum du nicht weißt, was du willst ist, dass dir diese Informationen fehlen.
Denke mal darüber nach:
\begin{itemize}
    \item Wie willst du heute entscheiden, ob dir der Job bei Bosch oder Siemens mehr Spaß macht?
\item Wie willst du heute entscheiden, ob du nach München oder Hamburg ziehen möchtest?
\item Wie willst du heute entscheiden, ob du mit Lisa oder Marie zusammen sein möchtest?
\item Wie willst du heute entscheiden, welchen Job du ,,für den Rest'' deines Lebens machen möchtest?
\end{itemize}

Du kannst das nicht mir Sicherheit beantworten. Denn die Antwort darauf liegt in der Zukunft.
Alles was du tun kannst ist, dich einer Entscheidung annähern. So, dass du mit gutem Gefühl eine Entscheidung triffst und die restliche Unsicherheit akzeptierst.

Sammele mehr Informationen für dein Gefühl und deinen Verstand. Es verbessert die Entscheidungsbasis auf der du die Frage ,,Was will ich wirklich?'' beantworten kannst.

\paragraph{Stürze Dich in neue Erfahrungen}
Höre auf zu Grübeln und stürze dich in die Aktion. Um herauszufinden, was du wirklich willst, musst du es fühlen! Niemand hat sich zu Hause auf dem Meditationskissen erleuchtet.
Der einzige Weg zu Klarheit führt über die Aktion, das Erleben!

\paragraph{Frage Menschen, die Du bewunderst}
Der einfachste Weg zu neuen Erkenntnissen ist: Frage Menschen die bereits erreicht haben, was du willst. 
Suche Menschen, die du bewunderst. Dann suche deren Nähe und lerne von ihnen.

\paragraph{Sag öfters ,,Ja''}
Du lehnst wahrscheinlich viele Möglichkeiten zu Lernen in deinem Alltag ab. Weil dein Kopf Dir sagt: ,,Darauf habe ich keine Lust.'' oder ,,Das kenne ich schon''.

Fakt ist aber: du hast keine Ahnung, wie die Dinge kommen werden. Und die besten Möglichkeiten entstehen oft aus den Dingen, von denen du es am wenigsten erwartet hättest.
Spring öfter über deinen Schatten und sage ,,Ja'' zu neuen Möglichkeiten. Auch wenn dein Gefühl eher ,,Nein'' sagen möchte.

\end{document}