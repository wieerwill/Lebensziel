 \documentclass[../Lebensziel.tex]{subfiles}
\graphicspath{{\subfix{../img/}}}
\begin{document}

\chapter{Das Vision Board}\thispagestyle{fancy}

Im Kapitel ,,Wer bin ich'' wurde das Vision Board bereits erwähnt. Das Vision Board ist ein Werkzeug, das dir helfen soll, deine Wünsche, Träume und Ziele darzustellen und dir in Erinnerung zu behalten.
Der Schwerpunkt bei einem Vision Board liegt in der Visualisierung deiner Gedanken. Je vielfältiger du dich einer Sache widmest, desto stärker bleibt sie in deinem Bewusstsein. Dabei übst du auch, mehr auf deine Ziele zu achten.

Der Sinn eines Vision Boards liegt darin, dir deiner Ziele, Wünsche und Ideen bewusst zu werden und sie präzise darzustellen. Die klare Zielformulierung in Form von Bild-Collagen hilft dir, deine Gedanken greifbar zu machen. Die tägliche Konfrontation mit deinem Vision Board unterstützt dich darin, dich deinen Zielen bewusst und manchmal unbewusst zu nähern und sie zu erreichen. Zudem stärkt das Vision Board deine Motivation.

\section{Ziele visualisieren}
Ein Vision Board ist eine Visualisierung deiner Wünsche und Ziele für dein Leben.
Das hilft dir aber wenig, wenn du es nicht nutzt.
Hier also die 5 Vorteile deines Vision Boards und warum du es verwenden solltest.

\subsection{Klarheit über deine Träume und Wünsche}
Indem du ein Vision Board erstellst, musst du dir die Fragen stellen: ,,Was will ich wirklich? Wie sieht meine perfekte Welt aus?''
Wünsche, die du im Alltag als ,,unerreichbar'', ,,unrealistisch'' oder ,,nicht so wichtig'' verwerfen würdest, treten zu Tage.
Durch dein Vision Board näherst Du dich deinen wahren Träume und Wünschen.

\subsection{Entscheidungshilfe}
Dein Vision Board hilft Dir (schwere) Entscheidungen zu treffen.
Denn nur, wenn du deine Träume kennst, kannst du entscheiden, was der ,,richtige'' Weg ist.
Bei jeder Entscheidung kannst du dich Fragen: Bringt mich diese Entscheidung näher zu meinen Wünschen oder entferne ich mich von ihnen?
Du weißt was Du willst und kannst dein Verhalten danach ausrichten.

\subsection{Mehr Fokus, mehr Aktion}
Plötzlich sind wieder zwei Wochen rum und du fragst dich, wo die Zeit geblieben ist.
Indem du dein Vision Board vor dir platzierst wirst du täglich an deine Lebensziele erinnert.

\subsection{Mehr Energie im Alltag}
Ein visuelles Vision Board hilft dir dabei, dich zu motivieren. Es hilft dir dabei, dich an deinen Sinn zu erinnern. An den tiefen, inneren Grund, warum du auf der Reise bist.
Dein Vision Board hilft dir, Kraft zu schöpfen und den nächsten Schritt zu gehen, auch wenn es schwer fällt. Denn du weißt jederzeit, warum du unterwegs bist.

\subsection{Inspiration für Schritte zu deiner Vision}
Dein Unterbewusstes findet automatisch Lösungen für Probleme, die es als wichtig erachtet.
Wenn du dein Hirn täglich mit der Vision deiner Wünsche fütterst, wird es automatisch Wege finden, diese Wünsche zu erreichen.
Plötzlich fallen dir Möglichkeiten auf, die dir vorher nicht in den Sinn kamen. Einfach, weil dein Hirn anfängt, unterbewusst nach Lösungen zu suchen.
Dein Vision Board ist ein Werkzeug deinem Unterbewusstsein eine klare Handlungsaufforderung zur Problemlösung zu geben.

\section{Wie sieht ein Vision Board aus?}
Es gibt kein perfektes Vision Board oder ,,das eine'' das für alle passt. Jedes Vision Board ist eine individuelle Collage deiner persönlichen Ziele, die du frei nach deinem Belieben gestalten kannst.
Es gibt aber Tipps wie dein Vision Board aussehen kann, um dir möglichst gut als Werkzeug dienlich zu sein. Lass dich auch von anderen Vision Boards, die du z.B. im Internet findest, inspirieren.

\subsection{Wähle deine 8 wichtigsten Lebensbereiche}
Ein Vision Board stellt eine allumfassende Vision deines perfekten Lebens dar. Teile daher dein Vision Board in die Lebensbereiche auf, die dir wichtig sind. Dadurch stellst du sicher, dass du nichts vergisst oder unter den Teppich kehrst.

Zur Inspiration sind hier ein paar Lebensbereiche
\begin{multicols}{2}
    \begin{itemize}
        \item Familie
        \item Freundschaften
        \item Sexualität
        \item Finanzen
        \item Berufung
        \item Karriere
        \item Freizeit, Hobbies und Erholung
        \item Spiritualität
        \item Gesundheit
        \item Liebesbeziehungen
        \item Persönliches Wachstum
        \item \dots
    \end{itemize}
\end{multicols}

Wähle bis zu 8 Bereiche aus. Mehr werden sonst zu unübersichtlich.

\begin{Form}
    \begin{enumerate}
        \item \TextField[width=12cm]{}
        \item \TextField[width=12cm]{}
        \item \TextField[width=12cm]{}
        \item \TextField[width=12cm]{}
        \item \TextField[width=12cm]{}
        \item \TextField[width=12cm]{}
        \item \TextField[width=12cm]{}
        \item \TextField[width=12cm]{}
    \end{enumerate}
\end{Form}

\subsection{Träume größer als jemals zuvor}
Träume frei, um dein Vision Board zu erstellen. Es gibt keine Grenzen. Schalte den inneren Kritiker bewusst ab.
Jetzt beginnst du, dein Vision Board mit Inhalt zu füllen.

\begin{itemize}
    \item Wähle einen Bereich deines Vision Boards.
    \item Dann nimm dir etwas Zeit und schließe die Augen. Frage dich selbst ,,Wenn ich alles haben kann, was ich will. Wie sähe mein perfektes Leben aus?''
    \item Nun beobachte. Welches Bild zeigt sich vor deinem inneren Auge? Nimm wahr, genieße und fülle dein Vision Board mit Skizzen, Worten und Ideen.
    \item Wiederhole diese Übung für die verbleibenden Lebensbereiche.
\end{itemize}

Wenn es dir anfangs schwer fällt, ein klares Bild zu sehen, dann helfen dir die folgenden Fragen vielleicht weiter.
\begin{multicols}{2}
    \begin{itemize}
        \item Wo befindest du dich?
        \item Was hörst du?
        \item Was riechst du?
        \item Was schmeckst du?
        \item Was fühlst du?
        \item Welche Menschen umgeben dich?
        \item Wie verhältst du dich?
    \end{itemize}
\end{multicols}

Achte auf dein Gefühl! Dein Gefühl ist das bestimmende Element. Verändere die Situation vor deinem inneren Auge so lange, bis du dich zufrieden, glücklich und geborgen fühlst. Es ist ein himmelweiter Unterschied zwischen dem, was wir denken, dass uns glücklich macht und dem, was es wirklich tut.

Wenn du genug gesehen und gespürt hast, dann öffne deine Augen und bringe zu Papier, was du gesehen hast. Das kann in Wörtern, Sätzen, Karikaturen, Bildern oder wilden ,,Krakeleien'' sein, die nur für dich Sinn ergeben. Es kann viel oder wenig sein. Wichtig ist, dass du eine lebendige Verbindung zu dem Gefühl und den Bilder spürst, die du eben erfahren und gesehen hast.

\subsection{Ergänze deine Zielkollage durch Bilder}
Ein Bild sagt mehr als tausend Worte. Und dein Gehirn denkt in Bildern, nicht in trockenen Worten oder Sätzen. Ergänze dein Vision Board mit aussagekräftigen Bildern, die dir gefallen. Damit wird dein Vision Board noch viel kraftvoller.
Du möchtest Bilder suchen, die genau das Gefühl ausdrücken, dass du vor deinem inneren Auge gespürt hast.

Wenn du das Bild anschaust, sollte in dir eine Lampe angehen, eine Wärme entstehen und eine Stimme sagen: ,,Ja genau, das ist es!''
Es sollte dich schlicht und einfach in das Gefühl von Zufriedenheit und Freude versetzten.

Am Anfang fällt es schwer, die richtigen Bilder zu finden. Es ist ein noch unbekanntes Gefühl und wird nur schwer wahrgenommen. Das ist in Ordnung. Wenn es dir ähnlich geht, dann suche einfach Bilder, die dir gefallen. Denke nicht groß darüber nach. Alles andere ergibt sich mit der Zeit.

\section{Anfängerfehler vermeiden}
Es gibt ein paar Anfängerfehler, die du vermeiden solltest. Folgende Punkte haben sich bewährt.
\begin{enumerate}
    \item Integriere alle Bereiche deines Lebens, die dir wichtig sind, in dein Vision Board. Zum Beispiel: Gesundheit, Familie, Sexualität, Beziehung, Finanzen, Berufung, Hobbies,\dots
    \item Schalte deinen inneren Kritiker ab. Dein Vision Board ist zum Träumen da. Nichts ist zu verrückt oder zu ,,unrealistisch''. Erlaube dir absolute Freiheit.
    \item Gönne dir Pausen. Intensive Selbstreflektion ist anstrengend. Machs dir gemütlich und lasse dir Zeit.
    \item Lass dich begeistern. Wenn du dein Vision Board anschaust, solltest du ein Gefühl von ,,Ja, will ich'' bekommen. das Gefühl ist wichtiger, als ,,Realismus''.
    \item Überarbeite dein Vision Board immer wieder, bis es sich stimmig anfühlt.
\end{enumerate}

\section{bewährte Vision Board Vorlagen}
\subsection{für Zielorientierte}
Diese Form des Vision Boards eignet sich besonders, wenn du deine Ziele über verschiedene Zeiträume visualisieren möchtest. Tabellarisch ordnest du dein Leben in seine unterschiedlichen, dir wichtigen, Bereiche. Die Zeilen stellen deine Ziele über einen Zeitraum von 6 Monaten, 12 Monaten und zwei Jahren dar. Du kannst diese Tabelle beliebig erweitern, um mehr Bereiche oder kleinere und größere Zeiträume.

\begin{center}
    \begin{tabular}{c|c|c|c|c}
                        & 6 Monate & 12 Monate & 2 Jahre & \dots \\\hline
        Lebensbereich 1 & \dots    & \dots     & \dots   & \dots \\\hline
        Lebensbereich 2 & \dots    & \dots     & \dots   & \dots
    \end{tabular}
\end{center}

\subsection{für Kreative}
Du hast keine Lust auf Struktur und möchtest deine Gedanken darstellen wie sie kommen? Das ist dieses Vision Board genau richtig für dich. Eine weiße Leinwand auf die du deine Wünsche und Visionen genau so sortieren kannst wie du möchtest.

\subsection{für Gewissenhafte}
Für einen strukturierten Überblick über alle deine Ziele, das du einfach nach deinen Prioritäten sortieren kannst.
In der Mitte finden deine persönlichen Werte Platz. Die Grundprinzipien, nach denen du in jedem Moment handeln möchtest.
Außen herum ist Platz für deine Visionen in jedem wichtigen Lebensbereich.

\begin{figure}[h]
    \centering
    \includegraphics[width=.8\linewidth]{img/Vision-Board-Beispiel-1.jpg}
    \caption{Gewissenhaftes Vision Board von \href{https://lebenistleidenschaft.de/vision-board-erstellen/}{lebenistleidenschaft.de} }
\end{figure}

\subsection{für Paare}
Ein Board für gemeinsame Vision mit deinem Partner.
Links stehst du und deine Wünsche. Rechts die deines Partners.
In der Mitte findet ihr Platz für gemeinsame Wünsche und Projekte.

\begin{figure}[h]
    \centering
    \includegraphics[width=.8\linewidth]{img/Vision-Board-Beispiel-4-Paare.jpg}
    \caption{Vision Board für Paare von \href{https://lebenistleidenschaft.de/vision-board-erstellen/}{lebenistleidenschaft.de} }
\end{figure}


\section{Die besten Orte für dein Vision Board}
Dein Vision Board repräsentiert deine Wünsche.
Platziere es prominent in deinem Leben und halte es im Fokus.
Dann hältst du auch deine Visionen im Fokus.

Bringe deinem Vision Board die gleiche Wertschätzung und Relevanz in deinem Leben entgegen, die du deinen Träumen entgegen bringen möchtest. Je prominenter du dein Vision Board in deiner Wohnung platzierst, desto kraftvoller wird es.

Die besten Plätze sind
\begin{itemize}
    \item Prominent im Wohnzimmer über dem Sofa
    \item Am Frühstückstisch, damit ich es jeden Morgen sehe
    \item An der Decke über meinem Bett. Der erste und letzte Blick an jedem Tag
    \item Innen an der Haustür. Damit ich mit der richtigen Energie in den Tag starte
\end{itemize}

\section{Ziele visualisieren mit deinem Vision Board}
So visualiserts du deine Ziele:
\begin{itemize}
    \item Setze dich vor dein Vision Board
    \item Wähle den Lebensbereich, den du visualisieren möchtest
    \item Versetze dich in die Situation, die du auf deinem Vision Board dargestellt hast. Spüre, wie es sich anfühlt, hättest du schon erreicht, was du auf deinem Visoin Board siehst
    \item Wiederhole die Übung am nächsten Tag
\end{itemize}

Beim Visualisieren geht es um das Fühlen. Je mehr du dich in die Situation hinein versetzt, desto stärker wirkt die Visualisierung. Stelle dir die Fragen, wie bei der Erstellung des Vision Boards: Was fühlst du? Was riechst du? Was siehst du? Welche körperlichen Empfindungen hast du? Welche Menschen umgeben dich? Je lebhafter du ein Bild vor deinem inneren Auge erschaffst, desto besser. 

Achte auch auf dein Gefühl und was sich seit der Erstellung deines Vision Boards verändert hat, haben sich deine Ziele verändert? Passe dein Vision Board kontinuierlich an, um dich und deine Ziele immer aktuell und im Gleichklang zu halten.

Nimm das Visualisieren deiner Ziele in deine Morgen- oder Abendroutine auf. Je regelmäßiger du visualisierst, desto stärker wird deine Motivation und Klarheit.

\end{document}